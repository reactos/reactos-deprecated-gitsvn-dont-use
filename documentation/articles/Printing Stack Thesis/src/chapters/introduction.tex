\chapter{Introduction}
\href{https://reactos.org}{ReactOS} is a modern desktop operating system entirely available under Open-Source licenses.
The Project is unique in the way that it aims for compatibility with all existing applications and drivers developed for Microsoft Windows.
This exclusive feature among free operating systems can make ReactOS an appealing alternative to the currently dominant desktop operating system.
By being distributed under Open-Source licenses, ReactOS can offer customizations and trustworthiness not possible with traditional closed-source systems.

Printing has become an essential feature of graphical desktop operating systems in the 1980 years.
The introduction of affordable Inkjet and Laser Printers around the same time have turned Personal Computers into Desktop Publishing machines able to produce high-quality documents \cite{leurs2013postscript}.
Since then, computers have mostly replaced traditional typewriters and typesetter systems.

Today, Printing is a self-evident ability of Personal Computers.
A desktop operating system is expected to detect and install connected Printers automatically as well as to provide intuitive options to manage and use them.
One of the common tasks of server operating systems is making a Printer available to multiple users over the network.
More recently, also smartphone and tablet operating systems have added support for Printing \cite{aosp2015kitkat}.
However, the ReactOS Operating System has not offered any support for Printing yet.

Operating system support for Printing also plays an important role in document exchange.
Creating a document in Adobe's popular \gls{PDF} is usually realized through a virtual Printer.
Such a Printer can be used as a destination from any text processing application just like a real Printer.


\section{Thesis Work}
This thesis presents an initial design and implementation of a Printing Stack for the ReactOS Operating System.

While several Open-Source Printing Systems already exist, none of them provides compatibility to the wide range of available Windows Printer Drivers.
On the other hand, hardware vendors often provide the most feature-rich drivers only for the Windows platform.
Therefore, the work does not build upon an existing Open-Source Printing System, but new components are developed from scratch.

In order to achieve ReactOS' goal of full compatibility to Microsoft Windows applications and drivers, the Windows Printing interfaces are analyzed in-depth.
Additionally, existing code of the ReactOS Operating System is examined to determine the extent of the required implementation work.
In a next step, a set of fundamental components is developed, which form an initial Printing Stack.
This system is able to transmit prepared RAW data in a Printer Control Language to a locally connected Printer.
However, the entire architecture is designed for a further extension by other datatypes at a later stage.

Evaluation of the written code occurs with the help of specific individual unit tests covering the implemented features.

Although Microsoft provides a publicly available documentation about their Printing implementation, it does not cover specific internals.
Therefore, a special emphasis has been put on commenting and documenting the resulting code to serve as a future reference.


\section{Special Thanks}
I would like to thank Univ. Prof. Dr.-Ing. Antonello~Monti and Dr. rer. nat. Stefan~Lankes for offering me the unique opportunity to make ReactOS part of my Bachelor thesis.
Special thanks also go to all members of the ReactOS Project who are maintaining this great Open-Source Project for several years.

\bigskip

\textit{\\
Colin Finck\\
Aachen, September 2015
}